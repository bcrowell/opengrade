\documentclass{opengrade_doc}
\begin{document}
\parafmt
\mytitle{OpenGrade} \\
\emph{Benjamin Crowell}\hfill{}\emph{www.lightandmatter.com}

\vspace{8mm}

\formatlikesection{Contents}

\noindent\large\begin{tabular}{ll}
\hline
Installing	& \pageref{installing} \\
Upgrading       & \pageref{upgrading} \\
Using the Graphical User Interface & \pageref{gui} \\
Goodies & \pageref{goodies} \\
Using the Text-Based User Interface & \pageref{textui} \\
Safety Features & \pageref{safety}\\
Passwords and Authentication & \pageref{authentication}\\
Import and Export of Other File Formats & \pageref{import-export} \\
Posting Grades on the Web & \pageref{web} \\
How Grades Are Calculated & \pageref{howcalc} \\
Format of Gradebook Files & \pageref{format} \\
Scripting & \pageref{scripting} \\
Future Improvements & \pageref{improve} \\
Bugs & \pageref{bugs} \\
Programming Stuff & \pageref{programming} \\
Format of Gradebook Files for OpenGrade2 & \pageref{old-format} \\
License	& \pageref{license} \\
\hline
\end{tabular}\normalsize


\mysection{Installing}\label{installing}

\mysubsection{Ubuntu Linux}
First, download the \verb@opengrade-...tar.gz@ file from \verb@lightandmatter.com@.

Open a terminal window.

If your browser didn't automatically unpack the file,
use a command such as \verb@cd Desktop@ to get into the directory in which the file was saved.
Then use the Unix command \verb@tar -zxvf opengrade...tar.gz@ to unpack it.

In the terminal window, do \verb@cd opengrade-...@ to move into the directory when opengrade's files are.

Install the depenencies by doing this:

\verb@sudo make ubuntu@

Finally, to install OpenGrade itself, do:

\verb@sudo make install@

You can now run the program with the command \verb@opengrade@. If you want to be able to do this
by clicking in a menu in GNOME, here's how to set that up. \emph{Right}-click on Applications in the top
menu bar. Click Edit Menus, Education, New Item. Fill in both Name: and Command: with "opengrade".

\mysubsection{Mandriva Linux}

OpenGrade's web page has information about how to locate up to date Mandriva RPM files.

\mysubsection{generic instructions for Linux}

If you're running Linux, but it's not one of the distributions described above,
the procedure for installing it should be similar to the one for Ubuntu, except that you'll want
to use your system's method for installing
the prerequisites.

The following perl modules are prerequisites:
\verb@Clone@,
\verb@Term::ReadKey@,
\verb@Date::Calc@,
\verb@Digest::SHA1@,
\verb@JSON@,
\verb@Tk@.
In addition, you'll need the ALSA \verb@aplay@ utility (needed only if you want sound).

As a last resort for installing the prerequisites, you can install all the perl modules from CPAN
by doing the command \verb@make depend@. This is a last resort, because CPAN probably won't play
nicely with your distro's packaging system.

\mysubsection{Windows}
I do not officially support OpenGrade on Windows, but my wife reported a few years back that it worked.

First you need to download Perl from \verb@http://www.activestate.com/@.
Starting from their homepage, follow the links for downloading ActivePerl.
They ask you for some registration information, which is optional, and then you download
ActivePerl. I've gotten reports that ActivePerl 5.8 breaks the installer
script for OpenGrade, so you should probably download and install 5.6
instead.\footnote{I found it confusing navigating their web site. Although
they change the site from time to time, here's what I had to do the last time
I went through the process. From the ActiveState home page, click on ActivePerl
under DOWNLOADS on the right side of the screen. Next, click on the red rectangle
that says Download. Click through the registration step using the Next button. Click
on MSI to download the Windows version of ActivePerl 5.6.}

Once you have ActivePerl installed, download
OpenGrade, and unpack the archive using WinZip or similar
software --- double-clicking on the file will probably work.
Reboot, make sure you have an active internet connection, and
then double-click on the file \verb@install_win.pl@.\footnote{If you don't
have privileges for administering your computer, you may find that installing
ActivePerl didn't correctly associate the .pl extension with ActivePerl. If this
is the case, then you'll have to run the installer by dragging it onto the file
C:$\backslash$Perl$\backslash$bin$\backslash$perl.exe, and the same drag-and-drop method will be required
in order to run OpenGrade later.} This file
will download and install a bunch of open-source software that's
required in order for OpenGrade to work; the last piece of software
it downloads, Tk, is large, and will take a long time to download over
a modem connection.
Wait until it's done, and reboot again.

 To run OpenGrade, double-click
on the file \verb@opengrade.pl@.

Starting with version 2.7.0, OpenGrade no longer has full support for sound on Windows.

\mysubsection{MacOS X}
The text-based user interface runs fine on MacOS X --- that's
the system I developed it on. I haven't tried getting the GUI
to run in X Windows on MacOS X.
You need to install the Developer Tools, and also X Windows, which is
now distributed with MacOS X.
Since MacOS X is Unix, the procedure for installing OpenGrade itself is similar
to the Linux procedure. 

\mysection{Upgrading}\label{upgrading}
The process of upgrading is similar to the process of installing, but if you're just
upgrading a minor version (e.g., 2.4.12 to 2.4.13), you shouldn't need to redo the part with
\verb@install_win@ (Windows) or \verb@get_dependecies_from_cpan.pl@ (Unix).
If you're upgrading a major version (e.g., 2.4.13 to 2.5.0), then you should redo that part. 
This applies in particular to the upgrade to versions 2.5\footnote{added dependency on Clone} and 
2.9.\footnote{added dependency on JSON}


\mysection{Using the Graphical User Interface}\label{gui}

\mysubsection{entering grades}
To see how OpenGrade works in everyday use, run the program as
described above, and open the file \verb@sample.gb@ that was
included among the files you downloaded. Enter \verb@secret@ for
the password. (OpenGrade uses passwords for authentication, but
there is no encryption. See page \pageref{authentication} for
more information on how this works.) Click on
exam 1 in the list on the left, and you should see something
like the window shown below.

\fig{doc_window}

On the right, you have a list of the students enrolled in the
class. Their scores on exam 1 have already been entered, but
you can change them now if you like. If you click on exam 2
from the list, you'll see their results from that exam.

Note the popup menu at the top of the column of scores.
With this menu, you can choose to see each student's average
test grade, or each student's overall grade in the course.
The two tests are the only assignments in the file right now,
so the exam averages are the same as the overall grades.

If you explore the other popup menu, above the list of assignments, you'll
see that there is another category, homework,
with no assignments in it. Let's add a new homework assignment.
Select the homework category from the popup menu, and then select
New Homework from the Assignments menu. Type 1 in the first space:
this assignment will be known as Homework 1. Type a number in the second
space for the maximum number of points possible on this assignment.
Click on OK. A blank column of scores pops up in the main window, and you can
enter the students' grades.

When entering grades, make sure you have your Num Lock turned on, or
else OpenGrade won't respond to numbers from the keypad. (It tries to
detect this situation, and will normally beep at you if it appens.)
When you hit
the Return key (on either the keypad or the main keyboard), the grade
will be entered, and the next student will automatically be selected.
(You can also use the down-arrow key as a synonym for Return, and the up-arrow
to select the preceding student.) This allows you to enter a list of
grades quickly if you already have them in alphabetical order.

If you have a stack of papers that is not in alphabetical order, and
you're entering all the grades, the quickest way to do it is to
use the keyboard to select each student. 
Here's how it works. Normally when you're
entering grades, the keyboard focus is in one of the little grade
boxes on the right-hand column of the spreadsheet. If you hit the
space bar, however, you'll see that now the student's name is highlighted
rather than the student's grade. Once the focus is in the left-hand
(names) column of the spreadsheet, you can use the keyboard to select
a different student by typing the first few letters
of her last name. 
The characters you've typed
will appear above the top of the list. Now hit the tab key, and
you can enter the grade. 
Note that if you keep on typing, the software assumes you're trying
to be more specific about what student you have in mind rather than
selecting a different student. For instance, if you have students
named Jarret, Johnson, and Olson, then typing \verb@jo@ means
that you're selecting Johnson; it doesn't mean you hit j, then
changed your mind and wanted to select Olson instead. If you hit
j and then change your mind and want to select Olson instead,
then you need to hit the backspace key to erase the j.
(The software will also forget about the letters you've typed
when you hit tab and take the keyboard focus away from the list
of students.)
If you get a unique match after typing the first three
letters, OpenGrade will make a little ``shk'' sound to let you know it worked.\footnote{This
is only if you have beeping enabled in your preferences, and
if you have the optional Audio::Data and Audio::Play Perl modules
installed.}
Once you get used to hearing it, you'll start to notice when you
don't hear it, which happens when you have two students with similar
last names; that's a warning that, e.g., you may have selected
Wilberforce (the first one in alphabetical order) when you wanted
Wilson.

When entering grades, the $+$ key is a shortcut for .5. 
For instance, a score of 3.5 points can be entered as $3+$.

\mysubsection{creating a new file}
To create a new gradebook file, choose New from the File menu
and fill in the form. You've created a new gradebook file, which
is empty. Now you need to get it set up correctly.

If you use a fixed set of grading standards, you can now enter
them by choosing Grading Standards from the Edit menu.

Next you'll probably want to set up all your categories of
assignments, such as exams, homework, and quizzes. To do this,
choose New Category from the Assignments menu. The first thing
you're asked for is a short name for the category.
It doesn't matter too much what you pick; in the graphical
user interface, you'll never even see it again. However, you will
see this short name if you ever use the text-based user interface,
or if you examine the gradebook file directly in an editor.
Just pick something short, like e for exams, or q for quizzes.
Make sure you don't use the same short name for two different
categories. Most of the rest of the inputs are fairly self-explanatory,
except for the space at the bottom for the weight of the category.
(To see this space, you may have to use the scrollbar on the right-hand
side of the window to scroll down to it.)
This is optional. If your student's grades are simply going to be
based on the sum of their points in the various categories, then
you should always leave this space blank. On the other hand, if you
want each category to have a certain weight, then you should enter
a weight for every category you create. See page
\pageref{weightedgrading} for more information on weighted grading.

Once you've created your categories, you can enter your students'
names using Add from the Students menu. If you already have your
roster in a text file, you may find it more convenient simply to
open your gradebook file in a text editor, paste in your roster,
and then put it in the form OpenGrade requires, as documented in
section \ref{format}.

\mysubsection{reports}
A variety of reports is available from the Reports menu.
The reports offered in this menu will depend on what student,
category, and assignment you've selected. When you choose
a report from the menu, it's displayed on the screen. You can
then hit OK if all you wanted to do was look at it, or Save
to save it to disk as a plain text file, e.g. for printing.

\mysection{Goodies}\label{goodies}
This section documents various features of OpenGrade that you may
or may not need. Almost everything in this section applies equally to the
GUI and the text-based interface.


\mysubsection{extra credit}
If an assignment is going to be for extra credit, you can simply
set it up with a maximum point value of zero, but enter scores that
are greater than zero. Normally, the software beeps at you and asks
for confirmation when you try to enter a grade that is higher than the
nominal maximum for the assignment. But when the maximum is zero,
it will accept nonzero values without complaining. 

Sometimes you have an assignment that is not completely extra credit,
but it may be possible for students to do something extra that gives
them more than the nominal maximum number of points. 
When you enter a score that's above the maximum, the software will
pop up a dialog box asking you if that's what you really want. To
avoid this dialog box, you can just type in the score with a trailing
letter x, e.g. \verb@101x@ on an assignment with a nominal maximum score
of 100.

\mysubsection{dropping a student}
When you drop a student, the student's grades will still be in
the gradebook file, so that if you need to, you can add her back
in later. In the GUI, you drop a student by clicking on her name
in the main window, and then selecting Drop from the
Students menu; to reinstate her, select Reinstate from the
same menu.

\mysubsection{scores that don't count}
You may wish to record scores that don't count at all in a student's
grade. (For instance, you might give a diagnostic test, or record whether
the student has completed some safety training.)
There is an option for this when you set up the category.

\mysubsection{dropping some scores}
When you're setting up categories, the software will ask you how many
scores to drop in this category. For instance, you could choose not
to count each student's two lowest homework scores.

\mysubsection{cloning a file}
The cloning feature allows you to make a new version of a gradebook file,
with all the students and grades deleted. This is useful when you're
starting a new semester, and want to get going with all the same
categories and grading standards you used last time.

\mysubsection{reconciling files}
If you enter grades both at home and at work, you have to keep synchronizing
your gradebook files between the two computers. If you forget to synchronize
them, you can end up with two inconsistent versions of the file, and it may
not be obvious what the differences are. To fix this, use Reconcile from the
File menu. It asks you to select two files, the first being the file you
want to change, and the second being the one from which you want to fold
in changes. Let's say these files are a.gb and b.gb, respectively. Basically
the result is to form the union of the two sets and put it in a.gb, i.e.,
after you're done, a.gb will contain any scores you entered in either of
the two files. It works the same way for students, categories, and assignments.
If a grade for a particular student is inconsistent in the two files, the program
will ask you whether to change a.gb so that it has the score from b.gb, or
leave a.gb alone. Grading standards, as well as properties of students, categories, and assignments are
left as in a.gb, so for example, if a student is marked as having been dropped
in a.gb but not in b.gb, the student is still marked as dropped. 

\mysubsection{marking periods}
Elementary and secondary school teachers often have the same students for a whole year,
but assign grades to them at the ends of certain marking periods, e.g., four quarters.
By default, marking periods are not enabled when you create a new gradebook file.
To enable them, choose Marking Periods from the Edit menu, and set the names and starting
dates of the periods.

\mysubsection{attendance}
To record attendance, set up a new category for it, and under Type choose attendance
rather then numerical. If you don't want attendance to affect students' grades,
choose that option when you set up the category. Each assignment in this category will now be an attendance
record for a particular day. A student's attendance score can be p for present,
a for absent, e for an excused absence, or t for tardy. If you can't remember these
codes, just try typing some other letter like z in the grade field, and you'll get
a message telling you what the possibilities are. In reports, attendance will be
shown as a percentage, with the convention that present, excused, and tardy all
count as 1, while absent counts as 0.

\mysubsection{printable rosters}
I often end up doing my grading when I don't have a computer handy, so I keep a printout
of my class roster in my binder and write grades on it. OpenGrade can print pretty
rosters for this purpose. To do this, you have to be running Linux, and you have to
have the free drawing program Inkscape installed. Do Roster from the Reports menu,
and choose Print Graphical Version Using Inkscape.

\mysection{Using the Text-Based User Interface}\label{textui}
OpenGrade has a text-based user interface that can be used instead of the GUI.
It only covers a small subset of GradeBook's functionality,
but I find that it can come in handy once in a while.
The user interface is menu-driven, and is fairly self-explanatory.

Often you'll need to select a particular student. Typically
you'll just enter the first few characters of the student's last name,
and hit return. As you go along, the software will display a list of
all students whose names match the characters you've typed so far. When
you see that there's only one student on the list, then you know you're
ready to hit return to select that student. Capitalization is ignored
here.

There is a time-saving trick that you can use when you have more than
one student with the same last name. For instance, my school has a large
Vietnamese population, and my classes usually have several students named
Nguyen. Suppose my class includes both Phong Nguyen and
Thu Nguyen. To select Phong Nguyen, I can type \verb@ng@, which narrows
it down to these two students, then hit the tab key, and then type \verb@p@.

\mysection{Safety Features}\label{safety}
Grades are valuable information, and OpenGrade has a whole
repertoire of methods to help you make sure you never lose any
of them. 

\mysubsection{auto-save}
Let's say you're editing a gradebook file called \verb@foo.gb@. 
OpenGrade will
save a copy of your work once per minute\footnote{If you're using the text-based
interface, it saves it every time you cycle through the main menu.}
under the
filename \verb@#foo.gb#@ (the same naming convention as the one
used by the Emacs editor). If your computer crashes in the middle
of an intense grade-recording session, you should be able to
recover most of your work from this file. (You may want to wait
one minute before rebooting, in case the program is still running and
there are changes it hasn't saved yet.)
The auto-save file is
deleted when you eventually save your file in the normal way,
so your directory should not get cluttered up with auto-save files.
Note that if you're recovering from a crash, you should
copy the auto-save file before starting OpenGrade again,
because the auto-save file will be overwritten when you open the
original file.

\mysubsection{auto-backup}
When you save the file \verb@foo.gb@, OpenGrade will
preserve the old version of the file under the name
\verb@foo.gb~@ (again, the name is based on the Emacs convention).
This takes care of the situation where you
save your file, and then realize you've messed something up.

\mysection{Passwords and Authentication}\label{authentication}
Every OpenGrade file has a digital watermark that is based on
the password you choose for the file. 
Suppose one of your students somehow gets access to your gradebook
files. Of course, this shouldn't happen, because you should store
your gradebook files on a computer account that nobody has access
to without logging in, but let's assume that you got up from your
desk without logging out, and a student has walked up and is trying
to tamper with your files.
If she tries to open one of your gradebook files using OpenGrade, it
will demand the password, and will refuse to open it without it.
(But don't leave your desk while you have a file open!)
If she opens the file using a text editor, she could start messing
around with its contents. However, when you get back to your computer
and open the file in OpenGrade, it will detect the fact that the
digital watermark is no longer valid. OpenGrade does not store your
password anywhere, so it assumes that the problem with the watermark
is because \emph{you} are an unauthorized person trying to tamper
with the file! It will refuse to open the file for you. This is
your warning that the file has been tampered with. OpenGrade will also
let you view a report that shows more information about the tampering,
e.g., which lines were tampered with, deleted, or added.

In everyday use, you should go by the following rules to protect your
files. Make frequent
backups and printouts. Don't store your gradebook files on a computer
that people can get access to without logging in. Never leave a file
open in OpenGrade if you're getting up from your desk. 
Use a good password: one that is at least eight characters long,
isn't a dictionary word, and contains both letters and digits.
Don't write your password down, and don't use passwords people can
guess, such as your license plate number, your kid's birthday, or
your pet's name. Don't type your password if someone can look at
your hands and see your keystrokes. Don't give anyone access to copies of your
gradebook file, because a determined hacker might be able to determine your
password by writing a program to guess millions of possibilities.

Here are some
real-life situations that can arise, and methods for dealing with
them:

\emph{You try to open the file, and OpenGrade doesn't accept your password.\/}
Don't panic! The most likely cause of this is that you've made a mistake
typing in your password. Just try again.

\emph{You've forgotten your password, and you need to edit the file.\/}
OpenGrade does not store your password anywhere, so there is no
way to get it back.
Use Change Password under the File menu to change the file's password.

\emph{You need to look at an old OpenGrade file from years
ago, and can't remember the password}
The easiest solution is to examine the file in a text editor. Otherwise,
change the password as described above.

\emph{A student tampers with your file using a text editor.\/}
Once you've convinced yourself that you didn't just make a mistake typing
in your password, here's what you should do. Click on the button to view
the report on the tampering, and save the report to a file.
Next, make a copy of
the file. You'll probably also want to print out both of these files.
Next, restore your gradebook file from your latest backup. (If you've been uploading
your grade reports to a server, then you should have a recent backup copy
there.) By examining the report on the tampering, you should be able to
repair some or all of the damage, and you may be able to figure out who
the tamperer might be, e.g., if all the changes were increases in a
particular student's scores.

\emph{A student changes the password of your file.\/}
This could happen if the student has read this documentation! You'll find
out that the file has been tampered with, because OpenGrade will refuse to
open it with your password. The procedures for handling this are
similar to the ones described in the preceding paragraph. However, you won't
be able to get a line-by-line report on the tampering.

\emph{This is too much security for you.
You don't want to be bothered with passwords.\/}
Just use a password with no characters in it. All you have to do is
click on the OK button whenever OpenGrade asks for a password.

\emph{This isn't enough security for you.\/}
Download Gnu Privacy Guard (also known as GPG or GnuPG), and use it
to encrypt your gradebook files.

\mysection{Import and Export of Other File Formats}\label{import-export}
OpenGrade has the ability to read and write several different file formats.

To export a file to some other format than the one it's already in, choose Export
from the File menu. After you export your file, you'll still have the one in the
original format open.

To convert a file from some other format to OpenGrade's format, open it, and then
export it to OpenGrade's format.

\mysubsection{OpenGrade (gb)}

OpenGrade's own native file format uses the file extension .gb. 

\mysubsection{OpenGrade version 2}

For backward-compatibility, you can save files in the format used by OpenGrade version 2.
This will result in the loss of data that are only supported by version 3 and later, e.g., attendance.

\mysubsection{Online Grades}

Opengrade can also write a format used by Online Grades, as described in the next
section.

\mysubsection{GradeL}

GradeL is an open-source gradebook for Linux which is similar to OpenGrade, but more
oriented toward primary and secondary education. OpenGrade can export files to
GradeL's .gdl format.

\mysubsection{spreadsheet}

To export OpenGrade data to a spreadsheet, choose Spreadsheet from the Reports menu.
This option is not really meant for exporting an entire gradebook to a huge spreadsheet,
but rather to get certain specific date out. You will be presented with a list of
choices for which grades you want to export. For example, suppose you wanted to
make a scatter chart to see how well students' homework grades were correlated with
their overall grades in the course. You would select homework and overall grade.
The spreadsheet would then have the students' names in the first column, and homework
and overall grades in the second and third columns. The data are written in a plain-text
format called CSV (comma-separated values), which can be read by spreadsheet programs
such as OpenOffice and Excel. You'll be shown the contents of the CSV file, and offered
the opportunity either to save it or to open it directly in OpenOffice Calc. When you
open it in OpenOffice Calc, Calc will present you with a dialog box asking you to choose
some options for how to interpret the CSV file. The one you need to change from its
default is "Separated by," which needs to be "comma."

\mysection{Posting Grades on the Web}\label{web}

This section describes OpenGrade's machinery
for posting grades on the web.
There are two different ways of doing this, both of which can
be accomplished using free and open-source software on the server
side. You can do it using Online Grades or Spotter.

\mysubsection{Using Online Grades}

Make sure the Online Grades plugin is enabled in the Preferences menu.
Once the plugin is enabled, you will have an Online Grades menu in OpenGrade.

Online Grades has a web site at \verb@http://www.onlinegrades.org@ that
describes how their system works.
They also have a demo you can try.
Their demo account has username

\verb|faculty@onlineNOSPAMgrades.org|

(Delete the NOSPAM.) The password is \verb@faculty@. OpenGrade is designed so that if you're using Online Grades,
your password for your gradebook file will be the same as your password for Online Grades.
Since you want to use the demo account, whose password is \verb@faculty@, you'll need to change
the password on your gradebook file to match that. Close any file you have open in OpenGrade,
and then do Change Password in the File menu; select your gradebook file, and set the password.
Now open that gradebook file, and do Settings from the Online Grades menu.
Under "Username or email address," put the email address above. The rest of the data is optional,
and can be set later, if you like. Now go back to the Online Grades menu and do
"Export Grades to Online Grades Format for Uploading." You will be prompted to save a file
with a filename ending in ".xml."

Now go to the demo page at \verb@http://www.onlinegrades.org/demo_info@.
Click on the button that says "Click Here to Automatically Login to the Online Grades
Faculty Center." Click on Submit File. Click the Browse button, and locate the
.xml file you just created. Click "Submit File to Online Grades." It should upload the file
and display a summary for you.

Note that the .xml file contains your password, which is a security risk if someone gets access
to your computer. You should probably get in the habit of deleting these files immediately
after uploading them.

\mysubsection{Using Spotter}

To post grades on the web, you'll need access to a server
where you can install files and CGI software yourself.
Install the \verb@ServerOG.cgi@ script
distributed with OpenGrade. To view their grades, students
will interact with a second CGI script, \verb@Spotter.cgi@, which
is distributed with the Spotter software, also on the
\verb@lightandmatter.com@ site. You will need to set up
a directory tree on your server as described in the Spotter
documentation, in the section titled ``Interfacing.''

Make sure the Spotter plugin is enabled in the Preferences menu.
Once the plugin is enabled, you will have a Spotter menu in OpenGrade.

Once this is all set up, you can upload grades by selecting
Post Grades from the Spotter menu.

\mysection{How Grades Are Calculated}\label{howcalc}
There may be situations where it's not obvious to you how the
software is calculating students' grades. This may happen when
you have categories in which you drop some low scores, or
when you have assignments that aren't due yet, but which some
students have already turned in. OpenGrade has definite mathematical
rules, documented in this section,
 for handling such situations. Although the rules are a little
complex, there is one general principle that will usually keep things
clear in your head: the number of possible points is always the same
for every student.
This principle has two advantages: (1) If we want to know who is
doing better in the course, student A or student B, we don't have to specify
whether we mean better in the sense of total points or better in percentage
terms. (2) It makes sense to compute statistics like averages and standard
deviations for the class as a whole, and these statistics can be computed
in terms of points, not just in terms of percentages. 

\mysubsection{dropping low scores}
 When dropping low
scores, we choose them so as to maximize the student's total points,
not the student's percentage. The maximum number of possible points in
the category is calculated by leaving out the assignments that have the
lowest number of points possible. Thus, the numbers deleted from the
list of scores do not necessarily correspond to the numbers deleted
from the list of possible points. If there is extra credit on some
assignments (i.e. the student has scores higher than the maximum),
then these extra credit points are skimmed off the top before any
of the calculations are carried out, and then added back on at the
end.

\mysubsection{due dates}
When you create a new assignment, the default is to leave the due
date blank; the software then assumes that the assignment has already
fallen due.

The only time you really need to enter a due date for an assignment is
when it's something that some students will turn in early. You only need
to enter the month and day, not the year; the year is figured out based
on the date you gave for the beginning of the term when you set up the file.
If an assignment isn't due yet, then it doesn't count in students' grades
at all. This is in keeping with the general principle that the number of points
possible is the same for all students. Thus, a student may turn in an important
assignment early, know she had a certain grade on it, and know that her
grade in the course is going to change a lot on the assignment's due date; but
the effect will not show up in the software until the due date.

An assignment is considered to have fallen due at the beginning of the day
it was due. For instance, if June 3 is the day you'll be turning in final
grades for your course, then you can set the due date on a term paper to
June 3, and it will be included in the grades you calculate on June 3.

\mysubsection{weighted grading}\label{weightedgrading}
You can set up a gradebook file for either weighted or point-total grading; this
is determined when you make your first category.

The weights do not have to add up to any particular number. For example, if
you had two categories, with weights of 80 and 20, the total would be
100, and they could be interpreted as percentages. On the other hand, you
could have used weights of 8 and 2, or 4 and 1, and the results would have
been the same.

With weighted
grading, if the total number of points possible in a particular category
is zero, then that category is treated as if it doesn't exist, and the
other categories then become more important. For example, exams could
count as 80 percent of your students' grades, but before the first exam
OpenGrade still needs to be able to calculate grades.
Because of this, there can be no such thing as an extra credit category
when you use weighted grading; an extra credit category is one in which
the number of points possible is always considered to be zero. If you
want to do extra credit with weighted grading, you need to make extra-credit
\emph{assignments} within specific categories.

\mysection{Format of Gradebook Files}\label{format}
This section describes the file format used by OpenGrade 3.
The old format used by OpenGrade 2 is described on page \pageref{old-format}.
The following is a listing of the sample.gb file distributed with OpenGrade,
with some lines replaced by \verb@...@ for brevity. Generically, the file
is in a format called JSON, defined at \verb@json.org@, which is a way of
representing arrays and dictionary hashes in a string. Arrays are represented
with square brackets, hashes with curly brackets. Although the files
written by OpenGrade are formatted in a fairly readable way, as shown in the
example below, the details of this formatting are not actually significant.
Whitespace outside of quotes is not significant, and the order of the key-value
pairs in hashes can be given in any order. In other words, if you're writing
an application that writes output in OpenGrade format, you don't have to
worry about doing any of these things in a particular way, but if you're
writing one that reads OpenGrade format, you shouldn't assume that it
will be formatted in a particular way.

\pagebreak

\begin{listing}{1}
{
"watermark":["SHA1","BehaQ3Z22YLptmbSwGzLw8dniPA"],
"data":
  {
    "class":    {
      "days":"MTWRF",
      "dir":"web_reports",
      "online_grades":{"cltext":"","course_code":"","section_number":"","term":""},
      "staff":"bcrowell",
      "standards":{"A":90,"B":80,"C":70,"D":60,"F":0},
      "term":"2009-1",
      "time":"",
      "title":"B. Crowell's physics class"
    },
    "types":    {
      "data":{
        "attendance":{"description":"attendance",
          "descriptions":{"a":"absent","e":"excused",
                   "p":"present","t":"tardy"},
          "order":["p","a","e","t"],
          "value":{"a":0,"e":1,"p":1,"t":1}},
        "numerical":{"description":"numerical"}},
      "order":["numerical","attendance"]
    },
    "category_order":["e","hw","att"],
    "categories":    {
      "att":{"catname":"attendance,attendance","type":"attendance"},
      "e":{"catname":"exam,exams"},
      "hw":{"catname":"homework,homework"}
    },
    "roster":    {
      "curie_marie":{},
      ...
      "goeppert-mayer_maria":{"dropped":"true"},
      ...
    },
    "assignment_order":["e.1","e.2","att.first_meeting"],
    "assignments":    {
      "att":      {
        "first_meeting":{"max":1,"name":"first meeting"}
      },
      "e":      {
        "1":{"max":100},
        "2":{"max":100}
      }
    },
    "grades":    {
      "att":      {
        "curie_marie":{"first_meeting":"p"},
        ...
      },
      "e":      {
        "curie_marie":{"1":97,"2":99},
         ...
      }
    }
  }}
\end{listing}

At the outermost level, the file is a hash with two keys, \verb@watermark@ and \verb@data@. A file does not
have to have a watermark; an unwatermarked file does not contain this outermost level of structure, which is
like an envelope surrounding the data inside. If you're writing a program that writes OpenGrade format,
you probably won't want to bother creating watermarks, and if the files are read into OpenGrade later,
OpenGrade will add the envelope itself. 

However, if you want to handle watermarks in your application, here's how they're computed.
Line 2 of the example gives the watermark, computed using the SHA1 hash algorithm.
The watermark is defined as follows. First, we take the bare data structure
without the envelope, and then we format it as a JSON string in a canonical way (which is different
from, and less readable than, the way it's formatted in the actual file).
The canonical format represents all hashes with their key-value pairs
in alphabetical order according to their keys. When a key-value pair has
a value that can be represented as a JSON integer (i.e., a string of digits
with no superfluous leading zeroes), the value is represented that way, not
as a string in quotes. All whitespace outside of quotes is eliminated.
On the front of this string, we concatenate the password. We then compute the
hash function (in this case SHA1) on this string, and encode it in base64,
an ascii representation of a bit string that is discussed in the Wikipedia
article Base64, and defined officially in RFC 1421.
Base64 strings are generally padded with \verb@=@ signs on the right in order
to make their lengths a multiple of 4 characters, and OpenGrade requires this.

Lines 5-14 give some per-class data.

Lines 15-24 define two types for grades: numerical and attendance.

Lines 26-30 define the grading categories: attendance, exams, and homework.
Each category has certain properties, including the mandatory \verb@catname@,
which gives singular and plural names for the assignments in that category.
The fact that none of the categories have a \verb@weight@ property implies
that this file uses straight points rather than weighted grading.

Lines 31-36 list the students in the class. In most cases, the form of the name
to be displayed is given implicitly by the database key, e.g., \verb@curie_marie@
will be displayed on the screen as Marie Curie. However, students can have
optional properties \verb@first@ and \verb@last@, which override this default.
On line 34, Marie Goeppert-Mayer is listed as having been dropped. Her grades are
still retained, and if she is reinstated in the class they will show up again.

Line 37 gives the order in which assignments will be displayed. A key like
\verb@e.1@ means the assignment named 1 in the category \verb@e@, i.e., exam 1.
Lines 38-46 give data about the assignments. Normally the name to be displayed
for an assignment is determined from its category and its name, so, e.g., exam
1 is displayed as \verb@exam 1@. However, line 40 shows that this default can
be overridden.

\mysection{Scripting}\label{scripting}

OpenGrade has some scripting features, which are probably mainly of interest to
me, for testing purposes.

Testing whether files are identical:

\verb@opengrade --identical a.gb b.gb >report || echo "not identical"@

Copying files, with parsing, error checking, and possible change of format:

\verb@opengrade --copy a.gb >/dev/null || echo "invalid file"@

\verb@opengrade --copy --output_format=old --output=b.gb a.gb@

\verb@opengrade --copy --output_format=json --output=b.gb a.gb@

By default, \verb@--copy@ strips the watermark.
If \verb@--copy@ is used with \verb@--authenticate@ and
\verb@--input_password=foo@, it will check the digital watermark on the input file
and propagate it to the output file.
(To use a null string for the input password, just omit the
option.)
Passwords can be added or changed by using \verb@--output_password=bar@.

To modify a file, use \verb@--copy@ with \verb@--modify@. For example,
the following command will delete the entire exam category ('e') from
the file a.gb:

\verb@opengrade --copy --modify='delete_category,["e"]' a.gb@

The argument to \verb@--modify@ consists of two parts, separated by a comma.
The first part is the name of a write method of the GradeBook class, defined
as part of the array \verb+@user_write_api_functions+ near the top of
GradeBook.pm. The second part is a JSON-formatted string representing
the arguments to be passed to that method.

\mysection{Future Improvements}\label{improve}
\mysubsection{general}
Features for elementary school teachers: class behavior, outside behavior,
letters to parents.
It would be cool to be able to make custom-formatted report cards; I think
the right way to do this is probably to use the facilities built into
a word-processor (OpenOffice?)
for merging data into a form.

All client-server transactions should be wrapped in a subroutine. The
subroutine should check whether the password works, and if not, prompt
for the right one. If not connected, should give an error message.
It should then reconcile the server and client:
server's e-mail overrides client's, and all other data from client
overrides server's: disabled accounts,...? Should also handle errors.
Need to deal with cases where the keys for the same student on the server
and client become inconsistent.

Downloading grades: Needs more flexibility
for scoring.

MyWords.pm has a mechanism for defaulting to English if the translation
in the preferred language isn't available. This should be generalized
so that the fallback isn't always English, and it should be in
Words, not MyWords.

Some server operations should be made more efficient. When uploading
grades, compression could be used. When listing work from the server,
the algorithm that runs on the server shouldn't read each file more
than once.

Make a test suite, with gb files that test as many features as possible,
and scripting.

Make sure unrecognized data in gb files is maintained across copies.

\mysubsection{GUI}
When creating a new assignment in the category of attendance, automatically supply a default
     name like "Monday, June 28," using the most recent class meeting.

When it says it's going to add a watermark, it doesn't actually do it unless you make some
other modification to the file to prompt it to write it to disk.

When you click on the go-away box of a dialog, the program doesn't want to quit after that.

There should be an easier way to look at your grading standards. You currently
can't look at them without going through a couple of steps as if you were
going to modify them.

A bunch of the menu items should have ``...''

When creating a new file, there needs to be a way to set whether weighted grading
will be used. (TermUI does this when you add the first category, but don't do it
that way in the GUI, too awkward. The decision really needs to be made when they
open the file.) 

Missing functionality: 
change order of assignments or categories.

The assignments listbox should start out positioned
at the end, not the beginning, when you select a new category. This seems to be
pretty difficult to do, however! Selecting and then unselecting the last item
doesn't work, and it doesn't have set() and get() methods like a Scrollbar widget.

The dialog box for creating a new file doesn't give all the
help that the text-based version does. It doesn't check 
 whether the days of the week are legal. Actually, I think
I can do this by building new types and options in to the Input class.

Should allow the user to view dropped students. Have a note in 
\verb@add_or_drop()@
about this.

Resizing needs to be handled better. Some widgets have hardcoded sizes, which
I'd like to avoid.

\mysubsection{text interface}

The terminal-based interface should figure out the width of your terminal
window, rather than defaulting to 80 columns.

Emacs-style tab completion is goofy in a case where one student's name
is the same as the beginning of another's (Ng and Nguyen) --- you need
to type the underbar!

In the statistics report, there is no user interface to say if you want
stats within a category, and the report routine doesn't support that,
either. The formatting of the report could be improved a lot.

When selecting a student, should be able to type in first name.
The display is also kind of messed up during the process of selecing
a student. And the underbar should be hidden.


Give some message when they make a choice that's
not on the menu, or not allowed because there is no open file, etc.
In main loop, can make hashes to control the behavior of the commands.

Menus should allow you to type multiple characters in a row and then
hit return so you go into a sub-menu. (Don't just take keystrokes without
hitting return, because some commands like ``revert'' are more than
one character.) Beep when they do an illegal command, and flush the
rest of the string out.

\mysection{Bugs}\label{bugs}
\mysubsection{general}
Some errors should be checked for when reading in the gradebook
file: two students with the same key; mangled syntax in parameter
lists; grades for a student who isn't on the roster; more than one
assignment in a single-assignment category.
This should be done in \verb@misc_initialization@.

The time zone correction in \verb@ServerOG.cgi@
when downloading grades may be incorrect if the client and
server are in different countries that have different definitions
of daylight savings time. It will also not work correctly if either
time zone is a non-integer. It might make more sense to have the time
zone as part of the Spotter file tree, so Spotter can do the time
zone correction when recording the grades.

On Windows, the preferences file is always saved in My Documents. Since
the My Documents folder could be on the C drive or the D drive, we search
those drives for the directory. This should work for must users, but if
it doesn't find My Documents on C or D, it just silently fails to create
the preferences file, which is a bug. I don't know much about Windows,
but I think preferences should probably go in Documents and Settings.

\mysubsection{terminal-based user interface}
The ``u'' command for uploading grades in the terminal-based interface is
untested, and may not actually work.

\mysubsection{GUI}
It's not possible to resize the window and cause the list of students to be
resized.

When starting the program from the command line, with a filename given as
an argument, it randomly sometimes makes the whole window too short. This
is probably some kind of race condition.

Sometimes a dialog box pops up as a small blank square instead of having
the right size, shape, and contents. This may be triggered by certain
WM actions such as tabbing the window with another window in Fluxbox,
or moving the window to a different virtual desktop. The only workaround
currently is to restart the program.

Certain network operations can freeze, e.g., uploading grades. This seems
to be because the network connection is flaking out. In this situation,
it should time out with an error message.

There is some GUI stuff relating to passwords in BrowserData::new().

The dialog box for setting standards is too tall when you use
ABCDF+- grading.

When you change the name of an assignment, the popup window that shows
the category above it is temporarily displayed incorrectly.

I wrote stuff into the Browser.pm package at the outside level, assuming
there would only ever be one window open. This needs to go into
BrowserWindow. Likewise, what happens with stuff like \verb@grades_queue()@,
 which
is inside a \verb@BEGIN{}@ block? Will its variable get properly initialized when I
make a new BrowserWindow object? Will each BrowserWindow object really have its
own \verb@%queue@?

\mysection{Programming Stuff}\label{programming}
\mysubsection{version history}
See freshmeat.net for a history of what changes have been made to the
code over time.

\mysubsection{internals}
You can use \verb@make internals@ to make the file \verb@internals.html@ from the
POD documentation.

\mysubsection{contributing}
If you'd like to make improvements to the software, please let me know!
You can find my current e-mail address at
\verb@www.lightandmatter.com/area4author.html@.
I haven't set up OpenGrade in a version control system such as Subversion or Git, but I would be happy to do so if
other people were really started doing a lot of work on the code.
For now, I think it would be easiest if contributors e-mailed me
patches. My current wish list/to-do list is given in the
Improvements and Bugs sections above. I would rather not make
any changes to the code that would add dependencies on
libraries that aren't available as Ubuntu packages, complicate the installation process, complicate the
user interface, or break cross-platform compatibility; if the feature
you want would violate one of these criteria, maybe it would be
better to contribute a separate utility to perform the function you
have in mind.

\mysection{Format of Gradebook Files for OpenGrade 2}\label{old-format}
This section describes the file format used by OpenGrade for versions before 3.0.
The following is a small example.

\begin{listing}{1}
class
  .title "B. Crowell's Physics 205/210, spring 2002"
  .staff "bcrowell"
  .days "MW"
  .time "13:30"
  .term "2002-1"
  .dir   "bcrowell/s2002/205"
  .standards "A:90","B:80","C:70","D:60","F:0" 

preferences
  .file "/home/bcrowell/.OpenGrade_prefs"

categories
  .e "catname:exam,exams","max:120"
  .hw "catname:homework,homework","drop:1"
  .q "catname:quiz,quizzes","ignore:true"

roster
  .newton_ike "id:1234"
  .einstein_al "id:5678","dropped:true"
  .oflaherty_karen "id:3333","last:O'Flaherty"

assignments
  .e.1
  .e.2 "max:135","name:midterm"

grades
  .newton_ike.e "1:119","2:135"
  .einstein_al.e "1:120"
  .oflaherty_karen.e "1:120","2:135"
\end{listing}

Lines 2-6 give general information about the class. The title is
placed at the top of web reports. The staff list gives usernames
of the instructors and TA's who may make changes to the gradebook.
The list of
days the class meets is
used to make it easier to enter certain dates, and likewise for
the line specifying the date on which the current
term began.  The days, time, and term fields are all optional, and
can be left as null strings.

If you're using Online Grades to post students' grades on the web,
the staff field is used
for the username, time for period, and title for the class's title.
In addition, there will be fields called \verb@online_grades_course_code@,
\verb@online_grades_section_number@, \verb@online_grades_term@, and \verb@online_grades_cltext@.
Some other settings for Online Grades will be stored in your preferences file.

If you're using Spotter as a web interface, then the \verb@dir@ field
is needed.
Some other settings for Spotter will be stored in your preferences file.

Line 7 gives a subdirectory of the directory \verb@cgi-bin/spotter@
in which this class's
grades will be stored; this is only needed if you're posting grades
on the web and letting students view them with Spotter.
Line 8 sets your grading standards for this class. On reports, these
symbols will be displayed along with the numerical scores. If you
don't want any grade symbols displayed, you can just leave the rest of
the line blank after \verb@.preferences@. The symbols are entirely
arbitrary.

Lines 10-11 tell where your preferences file is located. OpenGrade
tries to put this in the customary location for your operating system.

Lines 13-16 specify that there are three categories of graded work in this
class: exams, homework, and quizzes. The short tags \verb@e@, \verb@hw@, and \verb@q@
are used internally, and are also as headings in some tables.
The \verb@catname@ parameter
gives two forms of the name, which are usually singular and plural. The
first one is used in constructing strings like \verb@exam 1@ for use
in reports, and the second one is for headings in reports.
In line 14, we specify that exams are worth a maximum of 120 points by
default; this is optional. In line 15, the parameter \verb@drop:1@
says that we drop the lowest homework score. In line 16, \verb@ignore:true@
says that quizzes don't count toward the students' grades at all.
The category parameters \verb@ignore@ and \verb@max@ are just
defaults that are applied when new assignments are created, and these
defaults can be overridden for individual assignments.

Lines 18-21 give the roster. Normally the names are given simply as
database keys, in the format \verb@lastname_firstname@. When the name
is displayed, the first character of each name is capitalized. When this
is not the correct capitalization, or when the name contains punctuation,
the correct form can be indicated separately, as in line 21.
Line 20 indicates that Al Einstein has been dropped from the course,
so he will not show up on reports. However, his data are still in the
file, and he can be added back in at any time.

Lines 23-25 list the assignments that have been scored so far: two
exams. The first exam is worth 120 points, the default for the exam
category. The second exam is worth a little more, and instead of
being called ``exam 2,'' it will be labeled on resports as
``midterm.''

Lines 27-30 list the students' grades.

This listing is different from the OpenGrade files you'd typically see, because
it lacks authentication codes. Authentication adds some data to the file
which is preceded by a \verb@#@ sign wherever it occurs. Every line
has some authentication information at the end, and there are some
separate authentication lnes at the beginning and end of the file.
The format of the authentication codes is described in the source
code of the \verb@LineByLine@ module. If you cut and paste this
example into a file, and then try to open it with OpenGrade, it will
refuse because it doesn't have valid authentication codes. You would
have to do Change Password from the File menu to add authentication codes.

I haven't written a formal grammar for the file format, but here's some
practical information about some of the possible ambiguities.

Sections
are typically separated by a single blank line, but the number of blank
lines between sections is not significant, and will not be
preserved when OpenGrade reads a file and then writes it back out.
The order of the sections is arbitrary, but to help
make the files more readable to humans, it's recommended to start with
the following sections in this order:
class, preferences, categories, roster, assignments, grades.
Other sections will be added in the future, and
an application that reads and writes the format is responsible for
preserving any sections that it doesn't understand.

Although the authentication code at the end of each line
is marked with a pound sign, it's
not treated the same as comments are typically treated in Unix
configuration files: it's required, not optional; it's not
preserved verbatim when a file is read and rewritten; and it's
not free-form.
The remainder of the discussion applies to what's left of the line after
the authentication code has been checked and stripped off.
No significance is attached to leading or trailing
whitespace, but for human readability, each section should be indented.
Programs that read and write the format can and should indent their
output consistently, disregarding the indentation of the original input.
A line that ends with a comma continues onto the next
line as if there was no newline after the comma.

\mysection{License}\label{license}
OpenGrade is free software, and it comes with source code. It is
copyright 2002-2005 by Benjamin Crowell, is available under version
2 of the GPL license,

\verb@http://www.gnu.org/copyleft/gpl.html@.

This documentation is copyright 2002-2008 by
Benjamin Crowell, and is available under the Creative Commons
Attribution-ShareAlike 2.0 license, 

\verb@http://creativecommons.org/licenses/by-sa/2.0/@,

or, at your option, under version 2 of the GPL license.

\end{document}
